\section{Lecture 3 -- 18th September 2023}
Let us recall \Cref{prop:integral over iff finite generation as a module} with which we ended the previous lecture. 
\begin{proposition}[=\Cref{prop:integral over iff finite generation as a module}]
  Let $A\subseteq B$ be rings. The elements $b_{1},\dots,b_{n}\in B$ are integral over $A$ if and only if $A[b_{1},\dots,b_{n}]$ is finitely generated as an $A$-module. 
\end{proposition}
One can deduce the following corollaries. 
\begin{corollary}\label{corr: integral elements are subring}
  Let $A\subseteq B$ be rings. The elements of $B$ integral over $A$ form a subring of $B$. 
\end{corollary}
% Maybe edit this. 
\begin{proof}
  If $b_{1},b_{2}\in B$ are integral over $A$ then $b_{1}b_{2}, b_{1}+b_{2}\in A[b_{1},b_{2}]$. 
\end{proof}
\begin{corollary}
  Let $A\subseteq B\subseteq C$ be a tower of rings. If $B$ is integral over $A$ and $C$ is integral over $B$ then $C$ is integral over $A$. 
\end{corollary}
% Maybe edit this. 
\begin{proof}
  If $c_{i}$ generate $C$ over $B$ and the $b_{j}$ generate $B$ over $A$ then the $c_{i}b_{j}$ generate $C$ over $A$. 
\end{proof}
We now turn to extensions of Dedekind domains. More precisely, let $A$ be a Dedekind domain and $K$ its fraction field. Let $B$ be the integral closure of $A$ in $L$. 
$$% https://q.uiver.app/#q=WzAsNCxbMCwwLCJCIl0sWzAsMSwiQSJdLFsyLDAsIkwiXSxbMiwxLCJLIl0sWzEsMywiIiwyLHsic3R5bGUiOnsidGFpbCI6eyJuYW1lIjoiaG9vayIsInNpZGUiOiJ0b3AifX19XSxbMCwyLCIiLDIseyJzdHlsZSI6eyJ0YWlsIjp7Im5hbWUiOiJob29rIiwic2lkZSI6InRvcCJ9fX1dLFswLDFdLFsyLDNdXQ==
\begin{tikzcd}
	B && L \\
	A && K
	\arrow[hook, from=2-1, to=2-3]
	\arrow[hook, from=1-1, to=1-3]
	\arrow[from=1-1, to=2-1]
	\arrow[from=1-3, to=2-3]
\end{tikzcd}$$
Let us consdier some examples from the number field and function field case. 
\begin{example}
  For a cubic extension of fields $\QQ(\sqrt[3]{2})/\QQ$, both $\ZZ$ and $\ZZ[\sqrt[3]{2}]$ are Dedekind domains. 
  $$% https://q.uiver.app/#q=WzAsNCxbMCwwLCJaWltcXHNxcnRbM117Mn1dIl0sWzAsMSwiXFxaWiJdLFsyLDAsIlxcUVEoXFxzcXJ0WzNdezJ9KSJdLFsyLDEsIlxcUVEiXSxbMSwzLCIiLDIseyJzdHlsZSI6eyJ0YWlsIjp7Im5hbWUiOiJob29rIiwic2lkZSI6InRvcCJ9fX1dLFswLDIsIiIsMix7InN0eWxlIjp7InRhaWwiOnsibmFtZSI6Imhvb2siLCJzaWRlIjoidG9wIn19fV0sWzAsMV0sWzIsM11d
  \begin{tikzcd}
    {\ZZ[\sqrt[3]{2}]} && {\QQ(\sqrt[3]{2})} \\
    \ZZ && \QQ
    \arrow[hook, from=2-1, to=2-3]
    \arrow[hook, from=1-1, to=1-3]
    \arrow[from=1-1, to=2-1]
    \arrow[from=1-3, to=2-3]
  \end{tikzcd}$$
\end{example}
\begin{example}
  Fix some finite field $\FF_{q}$ of characteristic not 2. Let $A=\FF_{q}[t]$ with field of fractions $K=\FF_{q}(t)$. Let 
  $$L=\FF_{q}(t)[\sqrt{t^{3}+t+1}]=\FF_{q}(t)[s]/(s^{2}-t^{3}-t-1).$$
  One can verify that $[L:K]=2$ where $K$ are the functions on $\PP^{1}_{\FF_{q}}$ and $L$ are the functions on the elliptic curve $s^{2}=t^{3}+t+1$ (more often written $y^{2}=x^{3}+x+1$) and the extension of fields $L/K$ corresponds to the 2-1 map $E\to\PP^{1}_{\FF_{q}}$ from the elliptic curve to the projective line. The integral closure of $L$ in $A=\FF_{q}[t]$ is $B=\FF_{q}[t][\sqrt{t^{3}+t+1}]$. 
\end{example}
\begin{remark}
  One could also take $A=\FF_{q}[\frac{1}{t}]$ and the same extensions $L/K$. But now $\sqrt{t^{3}+t+1}$ is no longer integral over $A$. 
\end{remark}
\begin{proposition}
  If $A$ is a Dedekind domain with fraction field $K$ and $B$ the integral closure of $A$ in $L$ for $L/K$ a finite extention of fields, the ring $B$ is an integral domain with fraction field $L$. 
\end{proposition}
\begin{proof}
  The integral closure of $A$ in $L$ is a subring by \Cref{corr: integral elements are subring}. It remains to show that $B$ has fraction field $L$. 
\end{proof}
We want to show a stronger fact, that $B$ is in fact itself a Dedekind domain. This parallels what we have seen before. For $K$ a number field and $L/K$ a finite extension of fields, $\Ocal_{L}$ is also a Dedekind domain. We are seeking a proof that generalizes appropriately to non-separable extensions and the function field case. In this setting, we lose nice properties such as separability. But first, let us return to the simpler case of separable extensions. 
\\\\
To consider this case, we recall some notions from our first course in algebraic number theory. For an extension of fields $L/K$, multiplication by an element $x\in L$ gives a linear transformation of $K$-vector spaces $m_{x}:L\to L$ by $\alpha\mapsto x\alpha$. The linear transformation admits a trace which we denote $\Tr_{L/K}(x)=\mathrm{tr}(m_{x})$ and a norm $\Nm_{L/K}(x)=\det(m_{x})$. If further $L/K$ is a Galois extension, we can compute $\Tr_{L/K}(x)=\sum_{\sigma\in\Gal(L/K)}\sigma(x)$ and $\Nm_{L/K}(x)=\prod_{\sigma\in\Gal(L/K)}\sigma(x)$. Finally, recall that the $K$-bilinear form $\langle x,y\rangle=\Tr_{L/K}(xy)$ is nondegenerate if and only if $L/K$ is separable -- a bilinear form is nondegenerate if $\langle x,y\rangle=0$ for all $y$ then $x=0$. We can now prove the following. 
\begin{proposition}
  Let $A$ be a Dedekind domain with fraction field $K$ and $B$ the integral closure of $A$ in $L$ for $L/K$ a finite extention of fields. If further $L/K$ is separable, the ring $B$ is a finitely generated $A$-module (i.e. satisfies Serre's ``Hypothesis F''). 
\end{proposition}
This is \cite[Ch. 1, \S 4, Prop. 8]{Serre}. 
\begin{proof}
  For $x\in B$ we know it is integral over $A$ by hypothesis so $\Tr_{L/K}(x)\in A$ as it is an integral multiple of the coefficient of $x$ by \cite[\href{https://stacks.math.columbia.edu/tag/0BIH}{0BIH}]{stacks-project}. Let $\{e_{i}\}$ be a basis of $L$ over $K$. Without loss of generality, we can take $\{e_{i}\}\subseteq B$ by clearing denominators. Let $V$ be the free $A$-module spanned by the $\{e_{i}\}$. Using the trace map, we consider the dual module 
  $$V^{*}=\left\{x\in L|\Tr_{L/K}(xy)\in A, \forall y\in V\right\}.$$
  We have $V\subseteq B$ by inspection as the $\{e_{i}\}$ were chosen in $B$ and $B^{*}\subseteq V^{*}$ where $V^{*}$ is the free module spanned by the basis dual to $\{e_{i}\}$ with respect to the trace bilinear form the existence of which is given by separability. In particular, $B$ is an $A$-submodule of $V^{*}$ which was finitely generated by the finite dual basis. 
\end{proof}
To show that integral closures of Dedekind domains are Dedekind domains, however, will require some lemmata, some of which was assigned as homework. 
\begin{lemma}\label{lem: equality of primes in B}
  If $\wp\subseteq\wp'$ are prime ideals of $B$ and $\wp\cap A=\wp'\cap A$ then $\wp=\wp'$. 
\end{lemma}
\begin{proof}
  Consider the quotient $B/\wp$ and for $x\in\wp'\setminus\wp$ its image $\overline{x}\in B/\wp$. Quotients are homomorphisms of rings so
  $$\overline{x}^{n}+\overline{a_{n-1}}\cdot\overline{x}^{n-1}+\dots+\overline{a_{1}}\cdot\overline{x}+\overline{a_{0}}=0$$
  and rearranging we see that $a_{0}\in(\overline{x})\subseteq\overline{A}$, that is, $a_{0}\in\overline{\wp'\cap A}$ but not $\overline{\wp\cap A}$ a contradiction. 
\end{proof}
\begin{lemma}\label{lem:descending chain stabilizes}
  In a 0-dimensional Noetherian ring, any descending chain of ideals stabilizes. 
\end{lemma}
\begin{lemma}\label{lem: ring B0}
  Let $A$ be a Dedekind domain with fraction field $K$ and $B$ the integral closure of $A$ in $L$ for $L/K$ a finite extension of fields. Let $w_{1},\dots,w_{n}\in L$ be an $L$-basis fully contained in $B$ and denote $B_{0}=A[w_{1},\dots,w_{n}]$. If $a\in A$ then $B_{0}/aB_{0}$ is a 0-dimensional Noetherian ring. 
\end{lemma}
Granting the lemmata, we can prove the theorem. 
\begin{theorem}\label{thm: integral closure is Dedekind}
  Let $A$ be a Dedekind domain with fraction field $K$ and $B$ the integral closure of $A$ in $L$ for $L/K$ a finite extension of fields. The ring $B$ is a Dedekind domain. 
\end{theorem}
\begin{proof}
  We seek to show that $B$ is Noetherian, integrally closed and of Krull dimension 1. By construction, $B$ is integrally closed. Suppose to the contrary that $B$ is of Krull dimension more than 1 and we have a strict chain of prime ideals $\wp_{1}\subsetneq\wp_{2}\subsetneq_{3}$ of $B$. Contraposing \Cref{lem: equality of primes in B} and intersecting down to $A$ we have $\wp_{1}\cap A\subsetneq\wp_{2}\cap A\subsetneq\wp_{3}\cap A$, a contradiction as $A$ was Dedekind by assumption and of Krull dimension at most 1. \\\\
  It remains to show $B$ is Noetherian. Let $w_{1},\dots,w_{n}$ be a $K$-basis of $L$ contained in $B$ with each $w_{i}$ integral over $A$ and let $B_{0}=A[w_{1},\dots,w_{n}]$. By \Cref{prop:integral over iff finite generation as a module}, $B_{0}$ is a finitely generated $A$-module and therefore Noetherian. To show $B$ is itself Noetherian, we will show that any ideal $I\subseteq B$ is a finitely generated $B$-module for nonzero ideals $I$. We claim that for $a\in I\cap A$, $B/aB$ is a finitely generated $B$-module. Granting \Cref{lem: ring B0}, consider the decreasing sequence of ideals in $B_{0}$ given by $(a^{m}B\cap B_{0},aB_{0})$ for $m\geq 1$ which correspond to a descending sequence $\overline{(a^{m}B\cap B_{0})}$ in $B_{0}/aB_{0}$. By \Cref{lem: ring B0}, they stabilize at some $m\geq n$. For $\beta\in B$, we want to show that $\beta\in a^{-n}B_{0}+aB$. Take $h$ minimal such that $\beta$ can be expressed as $a^{-h}B_{0}+aB$. If $h\leq n$ we are done. Otherwise suppose to the contrary that $h>n$. Let $\beta=\frac{u}{a^{h}}+a\widetilde{u}$ for $u\in B_{0},\widetilde{u}\in B$. Rearranging this expression, we have $a^{h}(\beta-a\widetilde{u})$ so $u\in a^{h}B\cap B_{0}$ which lies in $I_{h}$ but $h>n$ contradicting the stabilization if ideals so $u\in I_{h}=I_{h-1}$ so $a^{h-1}\widetilde{u'}+au'$ so $u'\in B_{0},\widetilde{u'}\in B$ so $\beta=\frac{u'}{a^{h-1}}+a(\widetilde{u}+\widetilde{u'})$ contradicting minimality of $h$. So $B/aB\subseteq(a^{-n}B_{0}\cap aB)/aB$ and $B/aB$ is a finitely generated $B_{0}$-module. Then $B/aB$ is a finitely generated $A$ module and for an ideal $I$ containing $aB$ of $B$, $I/aB$ is a finitely genereated $A$-module in $B/aB$ with $A$ Noetherian, so $I$ is a finitely generated $B$-module. 
\end{proof}