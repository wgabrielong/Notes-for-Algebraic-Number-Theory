\section{Lecture 6 -- 27th September 2023}
We have previously discussed discrete valuation rings. Today we will focus on one important property, Hensel's lemma. 
\begin{lemma}[Hensel I]\label{lem: Hensel I}
  Let $K$ be a complete field with discrete valuation $v$ and $A$ its discrete valuation ring. If $f\in A[x]$ and $a_{0}\in A$ such that $\overline{a_{0}}$ is a simple root of $\overline{f}\pmod{(\pi)}$ then there is a unique root $a\in A$ of $f$ congruent to $\overline{a_{0}}\pmod{(\pi)}$. 
\end{lemma}
The lemma states that if $f$ has a solution modulo $(\pi)$, that solution can be lifted to a solution of $f$ in $A$ itself. This is a very strong statement where we make great use of the characteristics of the non-Archimedian metric. As opposed to the Archimedian setting, Newton's method works fantastically in the $p$-adic topology. 
\begin{proof}[Proof of \Cref{lem: Hensel I}]
  We will construct a Cauchy sequence $\{a_{n}\}_{n\in\NN}$ of elements in $A$ such that $f(a_{n})$ is a root mod $(\pi^{n+1})$ whose limit is $a$. We will then show that $a_{n}$ is unique mod $(\pi^{n+1})$ given the conditions above. 
  \\\\
  Suppose we have $a_{0}\in A$ such that $\overline{a_{0}}$ is a simple root of $f$ modulo $(\pi)$ and set $a_{n+1}$ to be $a_{n}-\frac{f(a_{n})}{f'(a_{n})}$ where $f'(x)$ is the formal derivative of the polynomial $f$. Note that $f'(a_{n})\not\equiv0\pmod{(\pi)}$. If so, we write $\overline{f}=(x-a_{0})\overline{g}$ mod $(\pi)$ and by the product rule we see that $\overline{f}'=\overline{g}+(x-\overline{a_{0}})\overline{g}$ so if $\overline{a_{0}}$ were a root of $\overline{f}'$, it would be a root of $\overline{g}$ implying that it was not a simple root of $f$. 
  \\\\
  Thus $v(f'(a_{0}))=0$, that is $f'(a_{0})\in A^{\times}$ and $a_{n+1}=a_{n}+\frac{f(a_{n})}{f'(a_{n})}\in A$. Recalling the binomial expansion, we write $f(x+y)=f(x)+y f'(x)+y^{2}A[x,y]$ so in our expression we have 
  $$f(a_{n+1})=f\left(a_{n}-\frac{f(a_{n})}{f'(a_{n})}\right)=f(a_{n})-\left(\frac{f(a_{n})}{f'(a_{n})}\right)f'(a_{n})+\dots$$
  where we have that the trailing terms have valuations at least 
  $$v\left(\left(\frac{f(a_{n})}{f'(a_{n})}\right)^{2}\right)\geq 2\cdot v(f(a_{n}))\geq 2(n+1)\geq n+2$$
  so $f(a_{n+1})\equiv 0\pmod{(\pi^{n+2})}$ demonstrating existence of the Cauchy sequence. 
  \\\\
  To demonstrate uniqueness, assume that $a_{n+1}$ is unqiue modulo $(\pi^{n+1})$ so if $f(a_{n}+h\pi^{n+1})=0$ then using our expansion above we have 
  $$f(a_{n})+h\pi^{n+1}f'(a_{n})+\dots$$
  where the trailing terms have valuation at least $n+2$ which determines $h$ modulo $(\pi)$ implying the uniqueness of $a_{n+1}$ modulo $(\pi^{n+2})$. This implies that the sequence is Cauchy, and taking $a=\lim_{n\to\infty}a_{n}$, we have $f(a)=0$ and $a$ is unique since each $a_{n}$ is determined uniquely modulo $(\pi^{n})$ for all $n$. 
\end{proof}
Note that Hensel's lemma comes in many forms and can be made more general in at least two ways which we now state. 
\begin{lemma}[Hensel, II]\label{lem: Hensel II}
  Let $f\in A[x]$ and $a_{0}\in A$ such that $\overline{a_{0}}$ is a simple root of $\overline{f}\pmod{(\pi)}$. If $v(f(a_{0}))>2\cdot v(f'(a_{0}))$ then there is a root $a\in A$ of $f(x)$. 
\end{lemma}
\begin{lemma}[Hensel, III]\label{lem: Hensel III}
  Let $f\in A[x]$ be a monic polynomial. If $\overline{f}$ factors as $g_{0}h_{0}$ for $g_{0},h_{0}\in (A/(\pi))[x]$ with $g_{0},h_{0}$ monic and relatively prime, then $f=gh$ for $g,h\in A[x]$ such that $\overline{g}=g_{0}, \overline{h}=h_{0}$. 
\end{lemma}
There is also a version that implies \Cref{lem: Hensel I,lem: Hensel II,lem: Hensel III} all together. Furthermore, $f$ need not be monic as stated in the version of \cite[Ch. 2, Thm. 4.6]{Neukirch}. Hensel's lemma is a very extensive statement stating that the vast majority of the time the factorization of polynomials in a complete DVR is determined by its factorization in the residue field. 
\begin{example}
  Consider $x^{p-1}-1\in \ZZ_{p}[x]$ a polynomial over the $p$-adic integers. THis factors into distinct linear factors $(\ZZ_{p}/p\ZZ_{p})\cong(\ZZ/p\ZZ)\cong\FF_{p}$. Applying \Cref{lem: Hensel II} repeatedly $p-2$ times, this implies that $x^{p-1}-1$ factors into distinct linear factors in $\ZZ_{p}$. That is, $\ZZ_{p}$ contains $p-1$ $(p-1)$th roots of unity. 
\end{example}
This gives a multiplicatively closed set of representatives for the residue classes for the $(p-1)$th roots of unity. Recall here that $\ZZ_{p}$ (uncountable) is much larger than the localizations $\ZZ_{(p)}$ (countable). Similarly if $A/(\pi)\cong\FF_{q}$ for $q$ some prime power then $K=K(A)$ has all $(q-1)$th roots of unity. 
\\\\
We are now able to define local fields. 
\begin{definition}[Local Field]\label{def: local field}
  Let $K$ be a complete field with discrete valuation $v$ and $A$ its discrete valuation ring. $K$ is a local field if and only if $|A/(\pi)|<\infty$, that is, it has finite residue field. 
\end{definition}
Local fields arise in two families: the mixed characteristic case and the equicharacteristic case. 
\begin{definition}[Mixed Characteristic Local Field]\label{def: mixed characteristic local field}
  Let $K$ be a local field and $A$ its discrete valuation ring with uniformizer $\pi$. $K$ is a mixed characteristic local field if $\mathrm{char}(K)=0$. 
\end{definition}
\begin{remark}
  In the setting of \Cref{def: mixed characteristic local field}, evidently $\mathrm{char}(A/(\pi))=p$ for some prime $p$ since $K$ is a local field. 
\end{remark}
In the setting of \Cref{def: mixed characteristic local field}, $K$ is characteristic 0 so its discrete valuation ring $A$ contains the integers $\ZZ$. So for $\mathrm{char}(A/(\pi))=p$, $p\in A$ where $v(p)\geq 1$ say $v(p)=e$. We thus write $p=\pi^{e}u$ for some unit $u\in A^{\times}$. This leads to the definition of the absolute ramification index. 
\begin{definition}[Absolute Ramification Index]\label{def: absolute ramification index}
  Let $K$ be a local field of characteristic 0 with discrete valuation ring $A$ and $\mathrm{char}(A/(\pi))=p$. The absolute ramification index is $v(p)$. 
\end{definition}
Note that in the setting above $p$ has positive valuation implying that $A$ contains the $p$-adic integers as well. Let $f=[(A/(\pi)):\ZZ/p\ZZ]$. One can show that $A$ is a free $\ZZ_{p}$-module of rank $ef$ and $[K:\QQ_{p}]=ef$ as well. This is an extension of Dedekind domains where $e$ is the exponent of $(\pi)$ lying over $(p)\subseteq\ZZ$.
\begin{example}
  Let $K/\QQ$ be a finite extension of fields unramified at $(p)$ and $\pfrak$ a prime of $\Ocal_{K}$ over $(p)$. The completion of $K_{\pfrak}$ with respect to the $\pfrak$-adic valuation $v_{\pfrak}$ is an absolutely unramified local field.
\end{example} 
This seems to suggest that absolutely unramified local fields are plentiful, but the following result shows otherwise. 
\begin{theorem}\label{thm: unique mixed characteristic extension}
  For every finite field $\FF_{q}$ of characteristic $p$, there is a complete local field $K$ of characteristic 0 which is absolutely unramified with residue field $\FF_{q}$. 
\end{theorem}
\begin{proof}[Proof of Uniqueness]
  Let $q=p^{f}$ and $\overline{\theta}$ generate $\FF_{q}$ over $\ZZ/p\ZZ$ whose existence is given by the primitive element theorem over finite fields with minimal polynomial $\overline{h}\in \FF_{p}[x]$ of degree $f$. Let $A,A'$ be complete DVRs of characteristic 0 absolutely unramified of characteristic 0 with residue fields $\FF_{q}$. Denoting $K,K'$ the fraction fields of $A,A'$, respectively, $[K:\QQ_{p}]=[K':\QQ_{p}]=f$. Since $\overline{h}$ is minimal, $\overline{\theta}$ is a simple root and applying \Cref{lem: Hensel I}, we can lift it to $\theta\in A$ whose minimal polynomial is $g\in\ZZ_{p}[x]$. Note that $\overline{g}(\overline{\theta})=0$ modulo $(\pi)$ so $\overline{h}|\overline{g}$ so the degree of $g$ is at least $f$, but $K,K'$ are degree $f$ extensions so the degree of $g$ is $f$ as well and $\overline{h}=\overline{g}$ and $K\cong \QQ_{p}[x]/(g(x))$. \\\\
  In $K'$ consider $g(x)\in A'[x]$ and note that $g$ splits into distinct linear factors over $\FF_{q}[x]$ since $g$ has root $\overline{\theta}\in\FF_{q}$ modulo $(\pi')$ where $\pi'$ is the uniformizer of the DVR $A'$ and $\FF_{q}/\FF_{p}$ is Galois. Applying \Cref{lem: Hensel II}, $g$ factors into distinct linear factors in $A'[x]$ so $K'$ contains a root of $g$ and $[K':\QQ_{p}]=\deg g$. But $g$ was irreducible over $\ZZ_{p}$ as $\overline{g}$ was irreducible over $\FF_{p}[x]$ so $K'\cong\QQ_{p}[x]/(g(x))$ giving the isomorphism.
\end{proof}
The proof of existence is \cite[Ch. II, \S 5, Thm. 3]{Serre}. 
\begin{example}
  Let $K/\QQ$ be a finite extension of fields with $\pfrak_{1},\pfrak_{2}$ over $(p)\subseteq\ZZ$. If $e_{1}=f_{\pfrak_{1}}=e_{2}=f_{\pfrak_{2}}=1$ then 
  $$[K_{\pfrak_{1}}:\QQ_{p}]=e_{1}f_{\pfrak_{1}}=1=e_{2}f_{\pfrak_{2}}=[K_{\pfrak_{2}}:\QQ_{p}]$$
  and $K_{\pfrak_{1}}\cong K_{\pfrak_{2}}\cong\QQ_{p}$. If $e_{1}=e_{2}=1$ and $f_{\pfrak_{1}}=f_{\pfrak_{2}}$ then $K_{\pfrak_{1}}\cong K_{\pfrak_{2}}$
\end{example}