\section{Lecture 5 -- 25th September 2023}
Let $A$ be a Dedekind domain with fraction field $K$ and $\pfrak\subseteq A$ a prime ideal. Note that $A$ and $A_{\pfrak}$ have the same fraction field $K$ which has all discrete valuations from the primes of $A$. We want to pass to a situation where we completely lose information about primes other than $\pfrak\subseteq A$. We do so via completion. 
\\\\
Let $K$ be a field with discrete valuation $v$ and discrete valuation ring $A$. The ring $A$ has an absolute value by taking $|x|=c^{v(x)}$ for $0<c<1$ setting $|0|=0$. Recall that by \Cref{def:valuation}, we have $|x+y|\leq|x|+|y|$, but the valuation with respect to a prime ideal satisfies a much stronger condition. 
\begin{definition}[Non-Archimedian Valuation]\label{def:nonarchimedian valuation}
  Let $K$ be a field. A non-Archimedian valuation on $K$ is a function $|\cdot|:K\to\RR$ such that:
  \begin{enumerate}[label=(\alph*)]
    \item $|x|\geq0$ for all $x\in K$ and $|x|=0$ if and only if $x=0$, 
    \item $|xy|=|x|\cdot|y|$, 
    \item and $|x+y|\leq\max\{|x|,|y|\}$. 
  \end{enumerate}
\end{definition}
\begin{remark}
  In Serre's text, non-Archimedian valuations are known as ultrametrics. 
\end{remark}
\begin{remark}
  The standard absolute values on number fields, $\RR$, and $\CC$ are Archimedian absolute values. 
\end{remark}
For any field $K$ with an absolute value $|\cdot|$, we have a completion $\widehat{K}$ with respect to the metric $d(x,y)=|x-y|$. These are equivalence classes of Cauchy sequences $\{a_{n}\}_{n\in\NN}$ with $a_{n}\in K$ such that for every $\varepsilon>0$, there exists $N\in \NN$ such that for $n,m\geq N$, $|a_{m}-a_{n}|<\varepsilon$. Two Cauchy sequences are equivalent if they get arbitrarily close together, as in the case of real analysis. We can define $\widehat{K}$ the completion of $K$ with elements Cauchy sequences with the pointwise operations. The additive and multiplicative identities are given by the constant sequences $\{0\}_{n\in\NN}$ and $\{1\}_{n\in\NN}$, respectively. Indeed, this is complete with respect to the absolute value $|\{a_{n}\}_{n\in\NN}|=\lim_{n\to\infty}|a_{n}|$. 
\begin{example}
  Let $K=\QQ$ and $|\cdot|$ the usual (Archimedian) absolute value. We have $\widehat{K}=\RR$. 
\end{example}
\begin{example}
  Let $|\cdot|$ be an absolute value arising from a discrete valuation $v$. We can define $\widehat{v}(x)$ such that for $x\in\widehat{K}$, $|x|=c^{\widehat{v}(x)}$ where $0<c<1$. Note that $c^{\ZZ}$ is discrete so we have $\overline{v}(x)\in\ZZ$ for all $x\in\widehat{K}$ so $\widehat{v}$ is a discrete valuation on $\widehat{K}$ and $\widehat{A}$ is the discrete valuation ring of $\widehat{K}$. In fact, $\widehat{A}$ is the topological closure of $A$ in $\widehat{K}$. 
\end{example}
\begin{definition}[Complete DVR]\label{def: complete DVR}
  Let $\widehat{K}$ be a field complete with respect to the metric induced by its valuation. The discrete valuation ring associated to $\widehat{K}$ is a complete discrete valuation ring. 
\end{definition}
One can also verify that $\widehat{A}$ can be realized as a projective limit 
$$\widehat{A}=\varprojlim_{n}A/(\pi^{n})=\left\{(r_{n})_{n\in\NN}:r_{n}\in A/(\pi^{n}) \text{ s.t. for }m>n, \overline{r_{m}}\equiv r_{n}\pmod{(\pi^{n})}\right\}.$$
Note that in the $p$-adic valuation, things are small if they are divisible by larger powers of $p$. This definition of the completion as inverse limit agrees with that notion. One can verify that $\pi$ is a uniformizer of $\widehat{A}$ and $\widehat{A}_{(\pi)}=\widehat{K}$. 
\begin{proposition}
  Let $S$ be a choice of representatives of $A/(\pi)$ in $A$. Every element of the completion $a\in\widehat{A}$ can be written unqiuely as a convergent power series $a=\sum_{n=0}^{\infty}s_{n}\pi^{n}$ with $s_{n}\in S$
  and every element of the complete field $x\in\widehat{K}$ can be written uniquely as a convergent Laurent series
  $$\sum_{n=m}^{\infty}s_{n}\pi^{n}, m\in\ZZ_{\leq0}.$$
\end{proposition}
\begin{example}
  Let $K=\QQ, A=\ZZ$ and $v$ the $p$-adic valuation with respect to some positive prime $p$. $\widehat{K}=\QQ_{p}$ the $p$-adic numbers with complete discrete valuation ring $\widehat{A}=\ZZ_{p}$, the $p$-adic integers. 
\end{example}
\begin{example}
  Let $K=\FF_{q}(t)$ and $v$ be the $t$-adic valuation, with respect to the prime ideal $(t)$. We have $\widehat{K}=\FF_{q}((t))$ the Laurent series in $t$ and $\widehat{A}=\FF_{q}[[t]]$ the power series in $t$. We can choose $S$ to be elements $\FF_{q}\subseteq\FF_{q}[t]$ so $S$ is an additive and multiplicative monoid and $A/(\pi)=\FF_{q}\subseteq K$. 
\end{example}
Let $A$ be a Dedekind domain with fraction field $K$, $\pfrak\subseteq A$ prime, and $v_{\pfrak}$ the valuation with respect to the prime ideal $\pfrak$. We write $K_{\pfrak}$ be the completion of $K$ with respect to the norm induced by the valuation $v_{\pfrak}$. We have $A_{\pfrak}\subseteq\widehat{A}\subseteq\widehat{K}$ taking $A_{\pfrak}$ to be constant Cauchy sequences. Note that $\widehat{A},\widehat{K}$ are much larger than $A_{\pfrak}$ in the sense that they have solutions to many more equations. These are examples of local fields. \\\\
We return to the setting of extensions. 
\begin{proposition}
  Let $K$ be a field with discrete valuation $v$ and valuation ring $A$ with $K$ complete with respect to the discrete valuation. If further $A$ is a Dedekind domain with fraction field $K$, $L/K$ a finite field extension, and $B$ the integral closure of $A$ in $L$, then $B$ is a discrete valuation ring, a free $A$-module of rank $[L:K]$, and $L$ is complete with respect to the metric from $B$'s valuation. 
\end{proposition}
\begin{proof}
  We show that $B$ is a DVR. The remainder of the proof can be found in \cite[Ch. 2, \S 2, Proposition 3]{Serre}. \\\\
  Let $\wp_{i}$ be the primes of $B$ with valuations $w_{i}$. Note that $\wp_{i}\cap A=(\pi)$ as $(\pi)$ is the only prime ideal of $A$ a DVR. By the equivalence of norms on a finite dimensional vector space over a complete field, each of the $w_{i}$s induces the same topology on $L$. Suppose to the contrary that $B$ is not a DVR, so $w_{1},w_{2}$ are not the same valuation. Pick some $x\in B$ such that $w_{1}(x)>0, w_{2}(x)=0$. In the topology induced by $w_{1}$ on $L$, $\lim_{n\to\infty}x^{n}=0$ but the limit of this sequence, if it exists, would not converge to 0 in the $w_{2}$ topology. This allows us to distinguish between the primes by their associated valuations, a contradiction, so there is only one nonzero prime of $B$ showing that $B$ is a DVR. 
\end{proof}
\begin{remark}
  Note that we required the hypothesis that $K$ complete with respect to the valuation. If $A$ were a Dedekind domain with more than one prime ideal, $A_{\pfrak}$ would be a DVR with the same fraction field as $A$. The integral closure $B$ of $A_{\pfrak}$ in $L$ of $L/K$ a finite extension of fields has primes $\wp_{i}|\pfrak$ and if there is more than one such $\wp_{i}$ then $B$ is not a DVR. 
\end{remark}
\begin{remark}
  In the case of a complete DVR, we need not assume $L/K$ is separable. We use Serre's Hypothesis F where $B$ is finitely generated as an $A$-module where $\sum e_{i}f_{i}=[L:K]$ but in the setup above $i=1$ so $[L:K]=ef$ and there is no splitting of primes. 
\end{remark}
This allows us to deduce the following connection between valuations on $K$ and $L$. 
\begin{corollary}
  For all $x\in L$, $w(x)=\frac{1}{f}v(\Nm_{L/K}(x))$. 
\end{corollary}
\begin{proof}
  This is \cite[Ch. 2, \S 2, Corr. 4]{Serre}. See the discussion following \cite[Ch. 2, \S 2, Proposition 3]{Serre}.
\end{proof}
We return to the setting of extensions of Dedekind domains. Let $A$ be a DVR with fraction field $K$, $L/K$ a finite extension of fields, and $B$ the integral closure of $A$ in $L$. The ring $B$ is a finitely generated $A$-module, but $K$ need note be complete with respect to the metric induced by the discrete valuation. Let $\pfrak$ be a nonzero prime ideal of $A$ and $\wp_{i}\subseteq B$ the primes lying over $\pfrak$. For $x\in K$, we have a valuation $v_{\wp_{i}}$ on $L$ by $e_{i}v_{\pfrak}(x)$, where we say $v_{\wp_{i}}$ prolongs the valuation $v_{\pfrak}$ to $L$ with index $e_{i}$. 
\begin{proposition}
  Let $A$ be a DVR with fraction field $K$, $\pfrak\subseteq A$ prime nonzero, $L/K$ a finite extension of fields, and $B$ the integral closure of $A$ in $L$. If $w$ is a valuation on $L$ prolonging $v_{\pfrak}$, then $w=v_{\wp}$ for some $\wp|\pfrak$. 
\end{proposition}
\begin{proof}
  Let $W$ be the DVR associated to $w$ with maximal ideal $\mfrak$, so $W$ is integrally closed with fraction field $L$. Since $w$ prolongs $v_{\pfrak}$, $A\subseteq W$ and since $W$ is integrally closed, $B\subseteq W$. Let $\wp=\mfrak\cap B$ and note $\wp\cap A=\pfrak$ so $B_{\wp}\subseteq W$ since the denominators outside $\wp$ were outside $\mfrak$. If $W$ contained something with negative $v_{\wp}$ valuation, then $W=L$ a contradiction, so $W=B_{\wp}$ from which it follows that the valuations agree. 
\end{proof}
Let $A$ be a Dedekind domain with fraction field $K$. For $\pfrak\subseteq A$, denote $v_{\pfrak}$ the associated valuation and $\widehat{K}$ the completion of $K$ with respect to the metric induced by $v_{\pfrak}$. Let $L/K$ be a finite extension of fields and $B$ the integral closure of $A$ in $L$. For each $\wp_{i}|\pfrak$, let $\widehat{L_{i}}$ be the completion of $L$ with respect to the metric $v_{\wp_{i}}$ and $\widehat{B_{i}}$ the valuation ring associated to $\widehat{L_{i}}$. Note that from $L$'s point of view, it can't see that all the $\wp_{i}$ lie over the same $\pfrak$. 
\begin{proposition}\label{prop: degree and splitting of completions}
  Let $A$ be a Dedekind domain with fraction field $K$, $\pfrak\subseteq A$ prime nonzero, $L/K$ finite, and $B$ the integral closure of $A$ in $L$. If $\widehat{L_{i}}$ is the completion of $L$ with respect to the metric $v_{\wp_{i}}$ for $\wp_{i}|\pfrak$ and $\widehat{B_{i}}$ the associated valuation rings then 
  \begin{enumerate}[label=(\alph*)]
    \item $[\widehat{L_{i}}:K]=e_{i}f_{\wp_{i}}$, 
    \item $L\otimes_{K}\widehat{K}=L\otimes_{K} K_{\pfrak}=\prod_{i}\widehat{L_{i}}$, 
    \item and $B\otimes_{A}\widehat{A}\cong\prod_{i}\widehat{B_{i}}$. 
  \end{enumerate}
\end{proposition}
\begin{proof}
  This is \cite[Ch. 2, \S 3, Thm. 1]{Serre} for (a) and (b), and \cite[Ch. 2, \S 3, Prop. 4]{Serre} for (c). 
\end{proof}
Of note here is (b), which states that the splitting of primes corresponds to the splitting of the algebra into fields. 
\begin{corollary}
  Assume the conditions of \Cref{prop: degree and splitting of completions}. If further $L/K$ is Galois and $D_{\wp_{i}}$ the decomposition group of $\wp_{i}|\pfrak$, then $\widehat{L_{i}}/\widehat{K}$ is Galois with Galois group $D_{\wp_{i}}$. 
\end{corollary}
\begin{proof}
  The action of $D_{\wp_{i}}$ extends by continuity to $\widehat{L_{i}}$ giving $e_{i}f_{\wp_{i}}=[\widehat{L_{i}}:\widehat{K}]$ automorphisms. 
\end{proof}